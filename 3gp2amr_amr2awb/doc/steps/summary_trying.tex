%!Tex Program = xelatex
\documentclass[11pt]{article}
\usepackage{fontspec}
\usepackage{xeCJK}
\usepackage{graphicx}
\setmainfont[Mapping=tex-text,BoldFont=WenQuanYi Zen Hei]{WenQuanYi Zen Hei}
\begin{document}
\section{基本功能验证总结及改进方向}
\subsection{音频播放}
    解决的技术点:
    \begin{itemize}
        \item wav文件解析
        \item rtp音频封包
        \item 发送数据至网络,客户端正确解码
    \end{itemize}

    效果演示方法:
        
        使用无误的wav音频文件,启动演示程序,使用vlc监听指定端口,
        可以听到正常放音。

    改进方向:
    \begin{itemize}
        \item wav文件兼容性,如多声道文件正确解码
    \end{itemize}

    性能测试:
    \begin{itemize}
        \item 800路全文件读取,并发,内存约800MB,cpu i5 2400 3.1G 4核平均占用约
              百分之70
    \end{itemize}
\subsection{视频播放}
    解决的技术点:
    \begin{itemize}
        \item 3gp文件中视频部分解析
        \item rtp视频封包
        \item 发送数据至网络,客户端正确解码
    \end{itemize}

    效果演示方法:
        
        使用无误的3gp视频文件,启动演示程序,使用ffplay监听指定端口,
        可以看到正常播放视频。

    改进方向:
    \begin{itemize}
        \item 3gp文件兼容性,如音视频在不同段落的3gp文件
    \end{itemize}
\subsection{音视频同时播放}
    解决的技术点:
    \begin{itemize}
        \item 确认3gp文件中无法存储mu-law音频数据
        \item 音视频数据同时发送至网络
    \end{itemize}

    效果演示方法:
        
        使用无误的3gp视频文件和wav音频文件,启动演示程序,使用ffplay和vlc监听
        指定端口,可以看到正常播放音视频。

\subsection{录音}
    解决的技术点:
    \begin{itemize}
        \item 音频数据接收
        \item rtp音频解包
        \item 音频存储为无格式文件,并可被正确解码播放,实现录音功能
    \end{itemize}

    效果演示方法:
        
    启动演示程序监听指定端口,对于收到的rtp音频数据存储成无格式音频文件,使用
    audacity导入,可以正常播放录制的音频。

\subsection{音频会议}
    解决的技术点:
    \begin{itemize}
        \item mu-law音频解码为14bit原始音频数据
        \item 双路原始音频数据混音
        \item 14bit原始音频数据压缩为mu-law音频
        \item 三路音频两两混合
        \item 网络接收三路音频数据,即时两两混合,并输出至网络,客户端接收
              解码正常,实现基本三方会议
    \end{itemize}

    效果演示方法:
        
        启动演示程序,向3路指定端口发送音频,同时监听3路指定端口,可以收到两两
        混合的3路音频。


    改进方向:
    \begin{itemize}
        \item mu-law与14bit原始数据转换使用查表法提升性能
        \item 多路音频混合具体方式选择:直接多路混合(整数,实数)、多级双
              路混合、外调函数库
    \end{itemize}
\section{其他改进方向}
    \begin{itemize}
        \item 性能优化,以及测试性能极限
        \item 高并发下的性能及稳定性测试,如多核压力不均,处理能力抖动,5ms内
              响应的百分比等
        \item erlang标准库的使用
        \item (高阶)其他高级音视频能力,如声效,视频叠加,图像变形等
        \item (高阶)其他音视频格式的支持,如amr,g.729,mp3,mp4等
    \end{itemize}
\end{document}
