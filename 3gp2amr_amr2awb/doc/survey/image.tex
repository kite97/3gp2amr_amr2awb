%!Tex Program = xelatex
\documentclass[11pt]{article}
\usepackage{fontspec}
\usepackage{xeCJK}
\setmainfont[Mapping=tex-text,BoldFont=WenQuanYi Zen Hei]{WenQuanYi Zen Hei}
\begin{document}
\section{数字图像}
\subsection{人眼对图像的感知}
    现实世界的物体有两种属性可以被人眼感知:颜色和形状。
颜色是人眼对于物体反射或发出的电磁波的反映,人眼可以感
受的电磁波是波长范围在380nm--760nm之间,颜色区间从红色
过度到紫色。形状是人眼中多个视细胞分别得到物体的一部分
而形成的点阵,再由大脑合成的一种感觉。客观世界的所有物
体均是由人眼反映成一幅各种颜色的点阵传入大脑,再由大脑
合成同时拥有颜色和形状的物体。

    虽然人眼可以识别成千上万种颜色,但人眼只有三种色觉
细胞,分别感觉红绿蓝三种色光。所以只向人眼传入三种纯色
的光线,大脑会误以为看到了真识的颜色,而非三种纯色。针
对人眼的此种性质,只需要三种颜色的光源,改变每种颜色光
线的强度,就可以向人眼发送成千上万种颜色了。
\subsection{图像的采集与显示}
    图像的采集设备完全模仿人眼的结构,记录现实光线对红
绿蓝三种纯色的亮度,将所有采集到的颜色点按照实际的位置
排列成点阵存储即可。采集设备记录到是每一个点称为一个像
素,每个像素均包含三种纯色的亮度值。

    显示设备本身即是一种模仿图像的设备,由若干像素组成
的一个点阵,每个像素均由三种原色的光源组成,所有像素同
时发光即可显示图像。对于特定图像的显示非常简单,只需将
记录到的所有的像素对应到显示设备上显示即可。
\subsection{图像的数字化与存储}
    将三种纯色的值进行数字化用于表示一种颜色,就可以将
颜色数字化。

    人眼对于三种纯色的敏感度不尽相同,对于红色的亮度变化
最为敏感,而对于绿色的亮度变化就不太敏感。但是将亮度的最
大值与最小值分成256份以上,则人眼均不能分辨任意两级之间
的变化。所以三种纯色的数值均可以表示成一个8位整数,一种
颜色可以用一组RGB值表示。

    将所有的颜色值按照点阵的位置存储为一张表,就可以实
现图像的数字化了。
\subsection{颜色空间}
    人眼对于真实颜色的处理,可以认为是获取一种颜色在三
种纯色上的分度值。反过来就可以说,三种纯色的值,可以表
示一种颜色。用三种纯色表示三个维度,则所有的颜色值均可
对应空间中的一点,空间中所有的颜色组成颜色空间。

    一般根据人眼对于纯色的感觉灵敏度,三种纯色可以取值
0--255。所以红绿蓝三纯色的颜色空间共可表示1677万种颜色。

    除三纯色的颜色表示法,颜色的分度还可以有其他的分解
方法,如YCbCr的表示方法。YCbCr中的Y表示光的流明值,Cb是
蓝色分量,Cr是红色分量,此种表示方法在显示时需要根据显
示设备的不同,使用不同的数值计算回三纯色并显示,并不方便,
但使用此种表示方法,图像将比较容易被压缩,节省存储空间。
\subsection{静态图像与运动图像}
    一般的图像仅指静态图像(still image)。将静态图像排列
在一起,快速的变换,利用人眼的视觉暂留,人眼会看到一幅
运动的图像(moving image)。人眼的视觉暂留时间约为24分之
1秒,所以每秒变换24幅以上的图像,人眼将会认为图像的变化
是连续的。在多媒体技术中,运动图像就是普遍意义上是视频(video)。
\subsection{静态图像的压缩}
    由于图像占用非常大的存储空间,为便于存储和交换,压缩
成为重要的技术手段。
   
    任何数据的压缩方式无外乎寻找一种编码方式,将数据中重覆
的部分用编码的方式更简短的方式表达。比如字符序列``aaaaabbbbb'',
设定编码方式为字符及重覆次数,可以用``a5b5''表达相同的意义,
而文本的长度明显简少了,即是``压缩''了。

    图像中,可以将每一个像素的数据类比为一个字符串,所有的像
素序列即可等同于一个字符序列,使用与
字符序列相同的处理方式,即可对图像信息进行压缩。但图像又与
字符序列有另一种明显的不同。字符序列间的重覆,是一种一维的
重覆,字符只需与前后两个方向结合寻找相同的字符。而图像的重覆
是二维的重覆,即是说图像中某一个小平面可能在图像中多次重覆,
但组成小平面的像素虽然在二维上相邻,在存储图像的数据中并不
相邻。所以需要开发针对图像的特别压缩算法才有更好的效果。
常用的png图像即是选取一种特别的lz77算法用于图像的压缩。

    使用上述的压缩算法对图像进行处理后,经过解压操作可以对
图像进行还原显示。由于解压后的数据与压缩前的的数据完全相同,
即是说数据量没有操损失,所以png选用的算法是一种``无损''压缩
算法。无损压缩对数据的压缩量是有限的,如果在压缩时损失一些
不必要的数据,但图像解压还原后给人的的感觉仍尽量保持一致,即
可在不影响图像整体质量的情况下减少大量的存储空间。

    jpeg对于图像的处理即使用有损压缩。jpeg先将原图像每个
像素的RGB值转换为YUV值,就是说换了一种颜色空间的表示方法,
与YCrCb类似,其中Y是亮度,U是色调,V是饱合度。由于人眼对于
亮度的变化更为敏感,而对色调及饱合度的变化相对差一些,所
以可以将图像中色调和饱合度相近的像素统一替换为相同的值,
这样图像中重覆的像素就明显增加了,再做压缩时,就可以获得
更高的压缩率。jpeg对YUV的做法是缩减取样,就是只对部分的
U和V值进行取样,进行处理。取样后,将图像以8*8大小为单位
变换为图像的频率值,再利用人眼对高频不敏感而减少高频相关的
数据,进一步减少最终数据。
\subsection{运动图像的压缩}
    运动图像实际存储相当于一组以时间为坐标的静态图像,
其中每一幅静态图像称之为一帧。所以,只要减少所有帧的大
小,就可以有效的减少运动图像的存储空间。m-jpeg技述即
是使用此种思路减少运动图像的存储空间。这种压缩方式称
为帧内压缩。

    运动图像的帧与帧之间的关系颇为密切,从整体上看,除
少数例外,大部分帧与之前和之后的帧的差别非常小。所以,
对于大多数帧,只需存储当前帧与前一帧的不同点,就可以通
过前一帧的数据得到当前帧的数据,极大的减少了存储空间。
同理,当前帧也可以存储与后一帧的差达到相同的目的。这种
压缩方法称为帧间压缩。对于帧间压缩,共有三种不同的帧最终
会被存储:I帧,是独立帧,本身含有所有的数据;P帧,需要
参考之前的I帧或P帧来计算当前帧;B帧,需要同时参考之前和
之后的帧才能计算当前帧。

    例:按存储及播放顺序如下

    IBBPBBPBBPIBBPBBPBBP

    解码时,先解出I,再解出第一个P,再解出两个B,再解出
第二个P,再解出两个B,以此类推,直至第二个I。第一I帧至
第二个I帧之前的序列,在mpeg编码中,被称作一个GOP(group of picture)。

\subsection{静态图像的容器格式与网络传输}
    图像的存储格式基本上均是基于块结构的流式存储,并不需
要获取全部数据才可以解码显示,除文件头外,可以获取一个块,
就解码显示一个块。本节说明静态图像的存储与网络专输,下节
说明动态图像相关内容。

    传统静态图像的存储是一个文件头,紧随图像的相关数据,
之后是文件结尾数据。如jpeg的容器jfif文件,将整个
文件分为SOI(start of image),APP0(图像基本信息),若干SOF
(图像数据)及一个EOI(end of image)几个部分。网络传输及解
码时,获得文件头后,顺序获得图像的数据,可以即时对图像的一
部分进行解码和显示。但jpeg的数据存储顺序是以行为单位从上至下,
从左至右进行存储,所以图像的显示顺序也将是从上至下,从左至右。
当文件很大或是网络状况不好,至使加载图像变慢时,图像将只能部
分的进行显示并等待余下数据。

    像jfif这样的存储方式不能很快的使人整体上对图像有一个把握,
使用体验并不好,png使用了另一种存储方式。

    png整体上与jfif一致,也是使用文件头与块的方式进行数据存储。
但是png先将图像分成8*8的小块,先存储所有块的最左上的像素,再存
储所有块的的四分之一大小的小块最左上的像素。具体存储顺序不在此
详述,但存储的方式是逐渐细化。这样在显示的过程中,图像的显示将
是从模糊到清晰,可以很快的总体上把握,体验相对较好。
\subsection{运动图像的数据流}
    运动图像因其结构相对静态图像比较复杂,观看时又有时间参量,
同时有适应网络化传输的客观需求,所以其数据流之中便存在一些结
构化信息,以便于满足需求。本节主要说明H.263格式的数据流的结
构信息。
  
    前文已经提到运动图像大体存储模式,即所有压缩后静态图像一
帧一帧的连续顺序存储,并无复杂结构,但每一帧的数据独立成段,
其内部的有结构的:

\begin{table}[htbp]
{ \small
\begin{tabular}{|l|l|l|l|l|l|l|l|l|l|l|l|l|}
\hline PSC&TR&PTYPE&PQUANT&CPM&PSBI&TRB&DBQUANT&PEI&GOBs&ESTUF&EOS&PSTUF \\ 
\hline
\end{tabular}
}
\end{table}
以下说明重点数据段,详细信息参见ITU-T的H.263的标准文档
\begin{enumerate}
    \item PSC: Picture Start Code,22位的同步头,值为0000 0000 0000 0000 1000 00,
        用于表求一帧的开始。
    \item TR:Temporal Reference。一般情况下每帧加一。
    \item PTYPE:类型信息,不定长。前两位为定值,10,表示开始代码仿真并表其为H.263
    数据。第三位,表是否分割屏
    幕。第四位,表是否为文档录像。第五位,表是否为全图冻结。第六位至第八位,表示图像
    格式,如011为CIF,100为4CIF,具体参见文档。第九位,图像编码类型,I帧为0,P帧为1。
    第十位至第十三位,略。如果第六位至第八位为111,则其为增量PTYPE,后续有其他数据位。
    \item GOB:Group of Block。每一帧图像按行分成若干个Block,这些Block共同组成GOB。
        其中,每一个Block可以分成若干MacroBlock,
    \item EOS:End of Sequence,22位结尾标志,值为0000 0000 0000 0000 1111 11。
\end{enumerate}


\subsection{运动图像的网络传输}
    本节说明H.263数据流通过RTP的方式在网络上传输。具本信息详见rfc2190。

    RTP首先通过SDP协商指定数据格式,指定rtpmap 34 H263/90000,即为
指定了视频为H.263。H.263经由RTP输一般选取采样率为90000Hz,此一次采样对
应一帧的图像。

    RTP承载H.263数据,共有ABC三种模式,在此仅说明最简单的A模式。
    
    因为
运动图像的数据量均偏大,所以在一次RTP传输中完成所有的数据传输是不可能的,
甚至不能完成一帧的传输,这就涉及到数据分段的问题。模式A规定,RTP整体以
帧为单位进行数据传输,当单帧亦不能经由一个RTP传输时,一帧的数据可以分成
多个RTP包传输。当一个RTP包是一帧的最后一个数据包时,RTP头的Mark位置为1,
其他情况置为0。RTP头中时间戳指定的是当前传输的帧所对应的采样偏移量,当
一帧被分成多个包传输时,时间戳的值相同。

除RTP头需要做部分修订外,rfc2190特别为H.263制订了头信息(H.263 payload 
header),其在RTP包中的结构如下:

{  \small
\begin{verbatim}

    0                   1                   2                   3
    0 1 2 3 4 5 6 7 8 9 0 1 2 3 4 5 6 7 8 9 0 1 2 3 4 5 6 7 8 9 0 1
   +-+-+-+-+-+-+-+-+-+-+-+-+-+-+-+-+-+-+-+-+-+-+-+-+-+-+-+-+-+-+-+-+
   |                 RTP header                                    |
   +-+-+-+-+-+-+-+-+-+-+-+-+-+-+-+-+-+-+-+-+-+-+-+-+-+-+-+-+-+-+-+-+
   |                 H.263 payload header                          |
   +-+-+-+-+-+-+-+-+-+-+-+-+-+-+-+-+-+-+-+-+-+-+-+-+-+-+-+-+-+-+-+-+
   |                 H.263 bitstream                               |
   +-+-+-+-+-+-+-+-+-+-+-+-+-+-+-+-+-+-+-+-+-+-+-+-+-+-+-+-+-+-+-+-+

模式A中的payload header的结构及意义如下:

    0                   1                   2                   3
    0 1 2 3 4 5 6 7 8 9 0 1 2 3 4 5 6 7 8 9 0 1 2 3 4 5 6 7 8 9 0 1
   +-+-+-+-+-+-+-+-+-+-+-+-+-+-+-+-+-+-+-+-+-+-+-+-+-+-+-+-+-+-+-+-+
   |F|P|SBIT |EBIT | SRC |I|U|S|A|R      |DBQ| TRB |    TR         |
   +-+-+-+-+-+-+-+-+-+-+-+-+-+-+-+-+-+-+-+-+-+-+-+-+-+-+-+-+-+-+-+-+

   F: 1 bit
   The flag bit indicates the mode of the payload header. F=0, mode A;
   F=1, mode B or mode C depending on P bit defined below.

   P: 1 bit
   Optional PB-frames mode as defined by the H.263 [4]. "0" implies
   normal I or P frame, "1" PB-frames. When F=1, P also indicates modes:
   mode B if P=0, mode C if P=1.

   SBIT: 3 bits
   Start bit position specifies number of most significant bits that
   shall be ignored in the first data byte.

   EBIT: 3 bits
   End bit position specifies number of least significant bits that
   shall be ignored in the last data byte.

   SRC : 3 bits
   Source format, bit 6,7 and 8 in PTYPE defined by H.263 [4], specifies
   the resolution of the current picture.

   I:  1 bit.
   Picture coding type, bit 9 in PTYPE defined by H.263[4], "0" is
   intra-coded, "1" is inter-coded.

   U: 1 bit
   Set to 1 if the Unrestricted Motion Vector option, bit 10 in PTYPE
   defined by H.263 [4] was set to 1 in the current picture header,
   otherwise 0.

   S: 1 bit
   Set to 1 if the Syntax-based Arithmetic Coding option, bit 11 in
   PTYPE defined by the H.263 [4] was set to 1 for current picture
   header, otherwise 0.

   A: 1 bit
   Set to 1 if the Advanced Prediction option, bit 12 in PTYPE defined
   by H.263 [4] was set to 1 for current picutre header, otherwise 0.

   R: 4 bits
   Reserved, must be set to zero.

   DBQ: 2 bits
   Differential quantization parameter used to calculate quantizer for
   the B frame based on quantizer for the P frame, when PB-frames option
   is used. The value should be the same as DBQUANT defined by H.263
   [4].  Set to zero if PB-frames option is not used.

   TRB: 3 bits
   Temporal Reference for the B frame as defined by H.263 [4]. Set to
   zero if PB-frames option is not used.

   TR: 8 bits
   Temporal Reference for the P frame as defined by H.263 [4]. Set to
   zero if the PB-frames option is not used.

\end{verbatim}
}

\subsection{运动图像的容器格式}
    运动图像的存储容器有多种格式,本节只针对3gp等电信相关的的容
器进行说明。
  
    运动图像的容器同样是基于块结构的,每个块都标明了块的名字,
长度,并包含相关数据。块中的数据也有可能包含子块。块的结构除了
利于网络上的流式传输之外,由于块的名字表明了块的意义,这样需
要添加新的功能块时,只要定义新的名字就可以了,非常容易,并有
很好的兼容性。

    3gp所使用的这种容器定义方法,最早源于苹果的mov文件,之后
被mpeg文档的第12部分定义为iso base media file format,3gp文件
又在此基础上扩充了若干新功能块。由于块的结构一致,所以识别
相关块名字,即可对3gp文件解析。

    3gp文件同时可以包含音频和运动图像,所以3gp文件可以说包含
的是视频。3gp文件中每一个块称为一个box,重要的块有moov和mdat等。
每一个块的均是由一个32位的整数表示长度,一个32位的4字节字符串表
示块的名称,其余的数据均是块的数据。数据部分可以是其他的子块。
以下仅介绍重要的块,具体请见mpeg文档第12部分。

    样例3gp文件结构分析如:

{  \small
\begin{verbatim}
Atom ftyp @ 0 of size: 20, ends @ 20
Atom mdat @ 20 of size: 925075, ends @ 925095
Atom moov @ 925095 of size: 7520, ends @ 932615
     Atom mvhd @ 925103 of size: 108, ends @ 925211
     Atom trak @ 925211 of size: 2990, ends @ 928201
         Atom tkhd @ 925219 of size: 92, ends @ 925311
         Atom mdia @ 925311 of size: 2890, ends @ 928201
             Atom mdhd @ 925319 of size: 32, ends @ 925351
             Atom hdlr @ 925351 of size: 45, ends @ 925396
             Atom minf @ 925396 of size: 2805, ends @ 928201
                 Atom vmhd @ 925404 of size: 20, ends @ 925424
                 Atom dinf @ 925424 of size: 36, ends @ 925460
                     Atom dref @ 925432 of size: 28, ends @ 925460
                 Atom stbl @ 925460 of size: 2741, ends @ 928201
                     Atom stsd @ 925468 of size: 117, ends @ 925585
                         Atom s263 @ 925484 of size: 101, ends @ 925585
                             Atom d263 @ 925570 of size: 15, ends @ 925585
                     Atom stts @ 925585 of size: 24, ends @ 925609
                     Atom stss @ 925609 of size: 120, ends @ 925729
                     Atom stsc @ 925729 of size: 28, ends @ 925757
                     Atom stsz @ 925757 of size: 1224, ends @ 926981
                     Atom stco @ 926981 of size: 1220, ends @ 928201
     Atom trak @ 928201 of size: 4406, ends @ 932607
         Atom tkhd @ 928209 of size: 92, ends @ 928301
         Atom mdia @ 928301 of size: 4306, ends @ 932607
             Atom mdhd @ 928309 of size: 32, ends @ 928341
             Atom hdlr @ 928341 of size: 45, ends @ 928386
             Atom minf @ 928386 of size: 4221, ends @ 932607
                 Atom smhd @ 928394 of size: 16, ends @ 928410
                 Atom dinf @ 928410 of size: 36, ends @ 928446
                     Atom dref @ 928418 of size: 28, ends @ 928446
                 Atom stbl @ 928446 of size: 4161, ends @ 932607
                     Atom stsd @ 928454 of size: 69, ends @ 928523
                         Atom samr @ 928470 of size: 53, ends @ 928523
                             Atom damr @ 928506 of size: 17, ends @ 928523
                     Atom stts @ 928523 of size: 24, ends @ 928547
                     Atom stsc @ 928547 of size: 28, ends @ 928575
                     Atom stsz @ 928575 of size: 20, ends @ 928595
                     Atom stco @ 928595 of size: 4012, ends @ 932607
     Atom udta @ 932607 of size: 8, ends @ 932615
\end{verbatim}
}
    
\begin{enumerate}
    \item ftyp:表示文件的的类型。
    \item mdat:表示3gp内的媒体数据,此样例中,音频图像共用一个数据
        段。音频数据是amr的媒体流,视频数据是H263的媒体流。
    \item moov:用于指示mdat中各数据的解释方法,如哪一段是音频,哪一段
        是图像,数据格式如何等。
    \item mvhd:说明整体信息。
    \item trak:轨道。一个轨道可能指明音频,也可能是图像和文字等,轨道
        内的数据以时间为参量,同时被使用。
    \item tkhd:每个轨道唯一,内部数据指明轨道的基本数据,如创建时间,轨
        道ID等。
    \item mdia:本身无内容,内部数据描述轨道的具体信息。
    \item mdhd:媒体格式无关的的媒体数据,包括创建时间,修改时间,时间伸缩,
        持续时间。
    \item hdlr:媒体的持有类型(handler)。
    \item minf:本身无内容,内部数据描述媒体信息。
    \item vmhd:视频媒体头。
    \item dinf:本身无内容,内部数据描述媒体数据。
    \item dref:数据参考。
    \item stbl:本身无内容,内部数据描述采样表。
    \item stsd:采样描述。
    \item s263:表示数据是H263。
    \item d263:内容为解码器信息。
    \item stts:用于解码的时间向采样转换。
    \item stss:同步采样表。
    \item stsc:采样至chunk的转换。
    \item stsz:采样的数据大小。
    \item stco:媒体数据chunk的入口,可有多个。
\end{enumerate}
    

\subsection{难点}
    高并发下的预处理。由于运动图像所占内存空间较大,进行大规模
缓存的压力较大。但是是不需要转码的情况下,所需的计算量比较小。
在高并发下,可以将运动图像以GOP为单位读入内存,发送完毕后,再
读取下一个GOP。
    
    RTP特别要求。RTP的Mark位在传输运动图像时有特殊用途。由于运动
图像每一帧都可能很大,无法在一个RTP包中传输完毕,所以,当RTP包是
当前帧的最后一个包时,mark位设置为1,其他时为0。
    
    音视频同步。音视频同步问题并不发生在服务器端。服务器仅能
以正确的时间间隔发送正确的数据。当接收数据时(录像),也仅能
做到以正确的顺序存储数据。

    丢帧。由于网络不稳定可能造成丢帧。当丢失B帧时,应假定丢失
帧与该帧之前一帧数据相同。当丢失P帧时,应假定与前一P帧或I帧相
同。当丢失I帧时,数据无法估计,应将整个GOP丢弃,以全黑填充。
\end{document}
