%!Tex Program = xelatex
\documentclass[11pt]{article}
\usepackage{fontspec}
\usepackage{xeCJK}
\setmainfont[Mapping=tex-text,BoldFont=WenQuanYi Zen Hei]{WenQuanYi Zen Hei}
\begin{document}
\section{收号功能}
\subsection{收号基本知识}
    收号有两种方式:带内(in-band)和带外(out-of-band)。带外收号是指使用音频数据通道
之外的数据通道来进行收号,如:使用SIP协议携带号码信息完成收号。带内收号指的是真接使用音频
数据通道进行号码信息的传输并完成收号。本次将开发的软板卡(以下称ocarina)不使用带外的方式,而使用带内的方式进行收号。

    使用与正常音频相同的数据通道传输号码有两种方式,一种是较为传统的方式,就是直接将号码信息
转换为音频数据(DTMF),板卡收到音频数据后对声音进行反向转换,进而得到号码信息,此种方式
不够稳定,比如RTP丢包,会对收号的正确性造成较大干扰;另一种是RFC2833新定义并由rfc4733修订的方式,
此方法新定义了一种RTP的数据承载类型——telephone-event,包含telephone-event的数据包除了描述了
与DTMF音频方式等价的信息,还增加了按键值,按键结束等相关信息,并且便于识别重传,
收号的稳定性大大提升。ocarina确定使用RFC2833新定义的方式进行收号。

    收号功能可以接收的号码是有一定范围限制的,一般只包含0-9这10个数字键及``*''号、``\#''号键,如果未作
协商,按RFC4733规定,默认的范围是0-15。
\subsection{承载类型协定及收号范围设定}
    RTP协议及RFC4733没有规定RTP承载类型具体值。表征一个RTP数据包是一个收号数据包类型值需要在
会话发起时,经由带外数据来协定。具体的方法之一是使用SDP协商,ocarina将使用此方法进行类型值协定。
    
    在协商过程中,SDP的数据项将会包含动态协定的承载类型,并在属性中明确协定的类型为telephone-event,
同时可以指明收号的范围。例如:

    m=audio 5166 RTP/AVP 0 8 101  (1)

    a=rtpmap:101 telephone-event/8000  (2)

    a=fmtp:101 0-9,10,11  (3)
    
    (1)指出使用协商的承载类型有0,8,101,其中0和8均是已经注册的承载类型,无须再
度说明,但101并非注册类型,需要动态协商。

    (2)明确101类型是telephone-event,同时指名采样率是8000Hz(采样率在时间戳生成
部分使用,后文详述)。

    (3)指明承载类型(event值)的范围是0-9这10个数字以及``*''(编号10)和``\#''
(编号11)。

\subsection{基本数据结构及数据填充说明}
   单一RTP数据包可以包含一个telephone-event数据包,共32bit,各字段长度及意义如下表
{  \small
\begin{verbatim}
 0                   1                   2                   3
 0 1 2 3 4 5 6 7 8 9 0 1 2 3 4 5 6 7 8 9 0 1 2 3 4 5 6 7 8 9 0 1
+-+-+-+-+-+-+-+-+-+-+-+-+-+-+-+-+-+-+-+-+-+-+-+-+-+-+-+-+-+-+-+-+
|     event     |E|R| volume    |          duration             |
+-+-+-+-+-+-+-+-+-+-+-+-+-+-+-+-+-+-+-+-+-+-+-+-+-+-+-+-+-+-+-+-+

            Figure 1: Payload Format for Named Events
\end{verbatim}
}
\begin{verbatim}
   event:8bit,存储event值。
   E:1bit,表示该数据包是否为一次按键的结束,0为非结尾
      数据包,1为结尾数据包。
   R:1bit,保留位,填0,处理时必须忽略。
   volume:6bit,按键音量,其值是省略负号的负整数,以db为单位。
   duration:8bit,按键自按下之时起,经历的采样次数,
      可以用来换算成按键持续时间。如采样率为8000Hz,
      按键时长为100ms,则取值为800。采样率通过带外的
      方式协商获得。
\end{verbatim}

   telephone-event发送时,第一个RTP包是起始包,需将RTP的mark位置为1。之后的是更新包,RTP的mark位为0,
同时RTP的时间戳与起始包相同,telephone-event的duration顺次递增。最后可能有若干个结尾包,telephone-event的
E位置为1,表示结束。所有的telephone-event的RTP包的序号均是前一个RTP的序号加1。若两个event连续发送,第一个
event最后的RTP包可以没有结尾标识,但第二个event的起始包,必须设置mark位。

\subsection{收号流程}
    收号是双向的,板卡可以从客户端收号,也可以向客户端发号,这里
单论从客户端收号的流程。

    ocarina的进程状态为主次模式。主为收取音频数据,次为收取号码。
    ocarina初始模式为主模式,当ocarina收到RTP包标识了mark位,并
且承载为telephone-event时,将自身置为次模式,并逐一收取后续
RTP包。按rfc4733 推荐方式,收号过程中丢弃所有的普通语音数据包。
在ocarina收到第一个telephone-event
数据包之后,立即处理收号结果,无需等待event结束。
在收到带有结尾标识的数据包之后,ocarina将模式转为主模式,并
且忽略后续的telephone-event包。
\subsection{技术难点}
    原则上,收号,收音频,以及收取其他数据之间地位是相等的。但是现阶段
ocarina仅支持收音和收号两种功能,于是设计ocarina的进程状态为主次模式。主
为收取音频数据,次为收取号码。技术的解决方案在此设计基础上提出。
    \begin{enumerate}
    \item 标准的选择。ocraina支持的收号方式在RFC2833中首先提
        出,并在RFC4733中重新进行修订。虽然RFC2833目前支持广
        泛,但某些条目不如RFC4733描述明确,所以尽量选择以
        RFC4733为标准,与RFC2833兼容。收号及录音时,完全兼容
        RFC2833,放音及发号时,除RFC4733明确描述的部分,与
        RFC2833兼容。
    \item 音频和telephone-event同时发送的处理策略。同时发送,
        指是的在同一个RTP流中,同一个event的几个数据之间,有音频
        的RTP数据包。RFC4733对于网关的建议中提到,网关在此种情况
        下应去掉音频数据包,仅发送event数据包。所以建议遇此种情况
        的选择是,不处理其中的音频数据,仅处理event数据。
    \item event值范围协商。event的值范围可以取0-255,常用的是0-11,如无
        协商则默认为0-15。由于初期软板卡并不需识别过多的event值, 过多协
        商也无意义,建议完全不协商范围,取默认。
    \item event重传的处理。event在发起后,会不断的发送更新信息,
        类似于同一事件的重传,同时,RFC2833及RFC4733建议在发送结尾时,
        重传三次带有结尾标识的包,以减少因为丢包造成收号无结尾的情况
        发生。如果ocarina在处理这些包时,收到结尾包后再处理收到的号
        码,会产生时延。建议在收到event数据包时,即时对收到的号进行处
        理,以加快响应 速度。
    \item 发起包与结尾包为同一RTP包。在主模式下遇此情形,则ocarina主模式
        无需改变,直接处理收到的号码。在次模式下遇此情形,处理号码后,次
        模式需转换为主模式。
    \item 数据包忽略。为增加系统稳定性,定义几种忽略数据包的情形:
        主模式下,收到非起始event的数据包;在次模式下,收到非event数据
        包。
    \item event结尾包丢失。收号时,在3次以上收到常规录音数据包后,认定为
        event结尾包丢失,系统上报收号事件并完成状态转换。
    \item duration取值越界。duration数据段的最大值大约可以表求一个event
        持继8s左右。超过8s的情形在RFC4733重新规定需要采取分片操作。
        由于超过8s的按键极少出现,建议ocarina暂时不支持超过8s的event。
    \end{enumerate}
    
\end{document}
