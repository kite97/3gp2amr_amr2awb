%!Tex Program = xelatex
\documentclass[11pt]{article}
\usepackage{fontspec}
\usepackage{xeCJK}
\usepackage{tabularx}
\setmainfont[Mapping=tex-text,BoldFont=WenQuanYi Zen Hei]{WenQuanYi Zen Hei}
\begin{document}
\section{整体录音流程}
    在未明确要求录音的情况下,ocarina静默的收取并丢弃所有收到的数据包。在收到
录音的指令后,开始接收收到的音频数据。在收到录音停止的指令后,根据mu-law的要求,
继续接收数据包0.125毫秒,如果配置为增加网络延时等待,则再多接收网络延时指定的时
长,如500ms,之后,将所有音频数据存储于指定位置的指定文件。
\section{关键技术点}
录音流程涉及到如下技术点:
    \begin{enumerate}
        \item 如何确认收到的数据的编码类型
        \item 数据何时存储
        \item 数据存储的格式
        \item 收到的数据包乱序
        \item 指定的存储的编码与收到的不相同
    \end{enumerate}
\section{技术点可选解决方案的对比及参考}
\subsection{如何确认收到的数据的编码类型}
    接收到的RTP包中,通过RTP包头的PT字段,可以确认RTP数据
部分的编码类型。

    RTP数据部分无法判断是否与PT字段相匹配。
\subsection{数据何时存储}
    数据存储的模式有两种:增量存储或内存完全缓存。增量存储指的是用一个固定的周期
将收到的数据追加到硬盘上,如:收到每一个RTP数据包之后或是每隔5秒钟这样的时间周
期。完全缓存指是的将收到的数据包全部缓存于内存之中,待收到结束指令后,一次性的写
入硬盘。

    增量存储可以有效的降低系统的内存压力,并可在系统突然失效时,有效的保存尽可能
多的数据,增加数据稳定性。而内存完全缓存则可以应对数据包乱序的问题,如:增量存储
的周期以时间为单位,并且设置过小时,乱序的数据包有可能还未能完全收到,就被写入硬
盘之中,新收到的数据如再重新修改已经写入硬盘的数据,将会影响系统性能,内存完全缓
存则可以在内存中完成整序而不会出现性能问题。

  性能与稳定性对比如下表:
\begin{table}[htbp]
\begin{tabularx}{350pt}{|l|X|X|}
\hline          & 全缓存                & 增量存储\\
\hline 内存压力 & 差,随时间线性增长    & 好,根据缓冲选取固定 \\
\hline 稳定性   & 差,故障时数据全部丢失& 好,仅丢失缓冲部分  \\
\hline 性能     & 好,数据全在内存中操作& 中,大部分数据在内存中操作  \\
\hline 磁盘压力 & 好,仅有一次顺序写入  & 差,每一次修改,磁盘增加两次随机寻址时间和一次写入时间  \\
\hline
\end{tabularx}
\end{table}

    解决方法是选择一个较长的时间间隔的增量存储。为应对突发的数据乱序,设置缓冲区
与内存数据区,存储流程变为:第一个时间段内收到的数据全部缓冲于内存,在第一个时间
间隔到达时,缓冲区内的数据进行解包等操作,将数据存于内存数据区,并开启新缓冲区,
此后如果发现有乱序的包,则修正内存数据区,当第二个间隔时间到达时,内存数据区写入
文件,缓冲区重复前面操作。

    IPS处理方法:暂无

    HMP处理方法:暂无
\subsection{容器格式的选择}
    容器格式主要有三种:无格式存储,定长数据容器和流式数据容器。无格式存储指将收到的原生数据
直接存储,无法指定数据的格式,解读时需要明确的知道数据的编码,但可以在仅增加尾部数据的情况下实现数据增量。
定长数据容器可以指定数据的格式,但是同时指定了容器内数据的长度,除了会产生容器数据量的上限之外,
每次修改文件,均需修改长度数据项,无法简单的只在尾部增加数据,修改复杂。
流式数据容器可以指定编码格式和持续的尾部数据增量,但处理流程复杂。
无格式存储属于私有格式的一种。私有格式可以有效阻断数据通用性,造成竞争对手的麻烦,同时也对自身造成
某些无法兼容问题。

    IPS未知

    HMP未知

    可选择的容器可以是定长的数据容器wav,也可以是无格式
数据容器pcm,或者是私有或是amr等流式数据格式。为简单起
见,当前建议使用无格式数据容器,将收到的数据按顺序存入
文件。文件名为<文件>.<编码>。如:msml指定存储为
test123.wav,文件内部编码格式为pcma,同时sdp协商使用的
数据流是pcmu格式的情况下,文件将暂存为test123.pcmu,待
录音结束后,交由后续工作进程转码并存储为test123.wav。

\subsection{包乱序的处理方法}
   数据包到达的顺序不正确, 可以通过重新整序来处理。假定数据包在发出后超过时间T未收
到即确定为丢包,则当前时间-T之前的所有数据均可正确整序。基于此,要求数据缓冲的时间
长于时间T即可。

    IPS处理方法:暂无

    HMP处理方法:暂无
\subsection{收到的数据编码与目标编码不同}
    收到的音频数据可以全部选暂存为无格式的数据文件,在录
音结束后,交由后续工作进程进行转码和及封装进指定的容器。
参见前文容器格式的选择。
\subsection{其他技术问题}
    \begin{enumerate}
        \item 如何判断丢包率。见RFC3550 A.3节。
        \item 检验承载数据的完整性。无法检验数据的完整性。
        \item 如何将数据按正确的顺序缓存。数据缓存在线性
            表中,通过第一个RTP包的序列号初始化偏移量,
            所有收到的数据存储于序列号减去偏移量对应的表
            中即可。
        \item SDP编码的选择。SDP会在协商时指定若干可用的
            编码,在交互过程中,RTP可以随时使用其中任一
            编码进行数据传输。详见RFC3264。
        \item 双方发送的RTP包的承载类型能否不一致。未在
            文档中读到明确的不允许。
        \item 网络收发对cpu的影响。tcp百兆网络跑满的情况
            下,没有对cpu造成压力。tcp全双工运行时,收发
            性能似乎与收发策略有关,比如全力收,间歇发,
            则会造成收取性能高于发送性能。udp大体应该与
            tcp相同。
        \item 静音。在G.711标准中,提及SID(Silence 
            Insertion Descriptor),可以用于发送静音。
            ocarina可以通过获取SID,来判断是否为静音。
            RTP静音包的传输参考RFC3389,其中规定静音包需
            通过CN(comfort noise)包传输。在RFC3551中规定,CN
            承载类型是13。
        \item DTMF。建议将DTMF当作普通语音录音。
    \end{enumerate}

\end{document}
