%!Tex Program = xelatex
\documentclass[11pt]{article}
\usepackage{fontspec}
\usepackage{xeCJK}
\setmainfont[Mapping=tex-text,BoldFont=WenQuanYi Zen Hei]{WenQuanYi Zen Hei}
\begin{document}
\section{会议功能}
\subsection{名词解释}
    \begin{enumerate}
        \item 进程:ocarina中使用的进程,是指一段可以单独
            运行的代码,不同于真实的系统进程,也不与真实
            的系统进程或线程相对应。进程的创建、销毁和调
            度等操作均可以在ocarina所在的系统进程的用户空
            间内完成。用于管理一个会议室所有资源的
            进程称为一个会议室管理进程,在不引起歧义的情况
            下,简称为“会议室”。
    \end{enumerate}
\subsection{会议基本流程}
    ocarina支持会议功能,实现一个会议的基本流程是:创建会议室,
会议室运行及响应请求,销毁会议室。

    创建会议室:ocarina收到开启一个会议的请求,创建一个会议室,
并初始化会议室基本信息,如:会议室可容纳人数。

    会议室运行及响应请求:会议室自身循环的接收所有与会人员发送
的数据(第一阶段仅限音频数据),会议室处理后,向目的与会人员发送。
同时,会议室响应ocarina发送的请求,对自身做出调整,如:增加与会人
员,调节与会人员的听说开关。

    销毁会议室:会议室响应ocarina发送的销毁指令,会议室释放
系统资源,并结束自身运行。

\subsection{系统会议功能}
    ocarina会议功能主要分成系统与会议室两个级别配合实现,本节说明系统
级别的功能要求。
    
    系统整体上负责会议室的基础数据获取和信令层面的交互。具体包括:

    \begin{enumerate}
        \item 接收会议室的创建请求,创建会议室。
        \item 接收会议室的销毁请求,销毁会议室。
        \item 接收会议室的查询请求,转达会议室,得到结果后
            回复信令。
        \item 接收会议室的修订请求,包括:增加删除与会人员,对指定
            人员的听说权限进行更改,转达会议室,得到结果后
            回复信令。
        \item 对全体会议室进行监控,包括:会议室总数,系统承载总人数,
            平均、最长、最短的会议室响应请求的时间,异常捕获处理和日
            志记录。其中日志记录包括自身和会议室的日志,会议室的
            日志均需发往系统,统一整理记录。
    \end{enumerate}
    
\subsection{会议室功能}
    ocarina会议功能主要分成系统与会议室两个级别配合实现,本节说明会议室
的功能要求。
    
    会议室负责会议功能主体实现。具体包括:
    \begin{enumerate}
        \item 增加与会人员,接收与会人员的音频数据。
        \item 删除与会人员,清除与会人员所占用资源。
        \item 对与会人员的听说权限进行设置。
        \item 将所有与会人员的音频进行混音,将合成后的音频发送至所有与会
            人员,同时确保与会人员不会听到自己的语音。
        \item 监控自身数据,上报系统。包括:数据总量,数据单位时间内流量,
            单个数据包的处理时长。
    \end{enumerate}
\subsection{技术难点}
    \begin{enumerate}
        \item 混音时延。混音时,为使语音合成数据有效,会产生时延。时延主
            要由两部分组成:数据时延和必要缓冲时延。
            
            数据时延指的是混音时,会议室需要将一段时间内的音频数据进行混
            合,而不是一次只混和同一个时间点上的若干采样。以两方混音为
            例,当混音时长定义为20ms时,混音系统会一次收集双方20ms的数
            据,将20ms的音频数据混合并输出。此时在理想情况下系统数据时延
            为20ms。数据时延与封包时长有关。
            
            必要缓冲时延指的是,混音系统为保证音质,增加的等待时长。以双
            方在网络上的混音为例,两方的每个数据包均包含20ms的音频数据,
            但有两点会使音频合成出现偏差:第一,双方的时间片分割的真实时
            间起止点不同,数据在真实时间对齐后,共有时间短于20ms;第二,
            双方相同时间点的数据包,到达并不同时,若以其中一方到达为准进
            行记时,会有部分情况丢失后到达的数据包。必要缓冲时延与网络抖
            动相关。

            混音时延的解决方案是加长数据缓冲时长,时长选择应为数据时延和
            必要缓冲时延综合计算的一个值,即为混音系统的时延。在此基础
            上,将混音双方的音频数据在真实时间上对齐,形成一个音频流,混
            音器在双方音频流上定时进行混音。

        \item 双方混音与多方混音。双方混音指仅混合两路音频,多方混音指同时
            将多路音频进行混合。双方所需同步方数少,所以实现简单,数据不易
            溢出,所需同步方数少,系统时延容易设置。多方混音整体上处理速度
            更快,并且可以控制混音的路数,比如只混音最大的三路音频。最初版本
            对速度要求并不严格,并且多方混音可以使用多级双方混音来完成,在
            对混音路数不做要求的情况下,建议第一阶段使用多级双方混音实现会
            议功能。

            混音的性能在产品化时需要尽力加强,所以混音的方式要认真选择。

        \item 回音消除。回音有两种,电学回声和声学回声,产生的原因均不在板
            卡。IPS板卡没有回音消除功能,dialogic可以消除电学回声,并且对
            系统性能会造成较大影响。建议不实现回音消除功能。

        \item 如何识别声音最大的几路音频。单就一个语音段计算声音最大非常
            简单,比较值即可。但在多方混音时简单计算的结果会使音量大约相等,
            排位在混音成员列表临界区的几路音频出现时而进入,时而退出,
            造成语音时断时续。建议在较大的时间片内取平均值,降低排位
            抖动频率。时间片时长基于人体听力的分辨时长,建议取值为100ms或其
            二倍200ms。

        \item 混音算法。公司拥有一个多方混音的专利算法,但其针对的是非常多
            方的混音。ocarina目前预计仅有10路音频,建议使用最简单的加和平均法。

        \item 中间格式的选择。混音最常遇到的问题是数据溢出。有多种算法来尽量
            在速度可以接受的情况下,将数据转换以保证合成过程中不产生溢出。
            混音中可以参考的开源软件有sox和ffmpeg。以目前的了解程度而言,最简单
            的保证不产生溢出的方法是牺牲速度,用更大的整数类型存储合成之后的值。
            混音遇到的另一个问题是舍入问题,由于整数除法没有小数,在多次合成后,
            依算法选择不同会产生音量逐渐变大或逐渐变小,并丧失部分细节。解决的
            方法是换用速度更慢但处理更为有效的实数表示声音。sox就使用实数表示声音。
            基于以上两点,由于初期对系统性能不做过多考虑,建议ocarina内部对于所有
            音频,全部转换为实数存储。

    \end{enumerate}
\end{document}
